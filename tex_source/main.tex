\documentclass[spanish,a4paper,11pt, openany]{scrbook}

% for images     
\usepackage{graphicx} 
 % Less detailed TOC
\setcounter{tocdepth}{3}   
% link outs in PDF
\usepackage{hyperref}
% for all characters
\usepackage[latin1]{inputenc}

%% Control the fonts and formatting used in the table of contents.
\usepackage[titles]{tocloft}
%% Aesthetic spacing redefines that look nicer to me than the defaults.
\setlength{\cftbeforepartskip}{1.5ex}
\setlength{\cftbeforechapskip}{2.5ex}
\setlength{\cftbeforesecskip}{0.5ex}

% German style of paragraph formatting, i.e. no indents.
\setlength{\parskip}{1.3ex plus 0.2ex minus 0.2ex}
\setlength{\parindent}{0pt}

% some layout    
\usepackage{geometry}
\geometry{tmargin=2.2cm,bmargin=2.4cm}

\begin{document}
\pagestyle{empty}
\author{Fran�ois Serra}
\title{Adaptation in genes, duplicates, families, functional
  modules and genomes}
\date{October 2011}
\maketitle
\newpage
\tableofcontents

\newpage
\tocloftpagestyle{fancy}

\chapter{Introduction}
\section{intro1}
\section{intro2}
\section{intro3}
\chapter{On structure of genomes}
\section{Introduction}
\section{Results and Discussion}
\section{Material and methods}
\chapter{Adaptation in genes, duplicated (and families)}
\section{Introduction}
\section{Results and Discussion}
\section{Material and methods}
\chapter{Gene set Selection analysis}
\section{Introduction}
\section{Results and Discussion}
\section{Material and methods}
\subsection{Dataset}
\subsubsection{5 mammals}
\subsubsection{6 \textit{Drosophila}}
\chapter{Genome neutrality}
\section{Introduction}
\section{Results and Discussion}
\section{Material and methods}
\section{open on colocalization \rightarrow not random}
\chapter{Tools, programs, methods}
\section{ETE-evol plugin}
\section{Phylemon}
\section{Ecolopy}
\section{Isoform selection?}
\chapter{Conclusions}

\cite{Kosiol2008a}

% %%% introduction.tex --- 

%% Author: garamonfok@gros
%% Version: $Id: introduction.tex,v 0.0 2011/10/09 18:39:32 garamonfok Exp$

\section{What is DNA? How genes rose?}
\section{Definition of neutrality}
\subsection{Neutrality in modularity}

Explanation of protein networks by to parameters probability of edge deletion $\delta$ and probability of link creation $\alpha$ after a single gene duplication \cite{Sole2008}

\section{Life in DNA, from genes to repetitive elements.}
\section{Adaptive changes to evolutionary speed}
\section{Evolution, and the detection at molecular level}
\section{Grouping genes and finding evolutionary patterns}


%%% Local Variables: 
%%% mode: latex
%%% TeX-master: "../../master"
%%% End: 

% %%% Local Variables: 
%%% mode: latex
%%% TeX-master: "../main"
%%% End: 
%%% introduction.tex --- 

%% Author: garamonfok@gros
%% Version: $Id: introduction.tex,v 0.0 2011/10/09 18:39:32 garamonfok Exp$

\part{Comparative genomics and adaptation}
\label{part1}

\tableofcontents

\clearpage
(TeX-add-style-hook "content_example"
 (lambda ()
    (LaTeX-add-labels
     "chap:one"
     "fig:jws"
     "tab:asdfasdf")))


%%% Local Variables: 
%%% mode: latex
%%% TeX-master: "../main"
%%% End: 

\chapter{Random-like structure of DNA}
\label{chap:two}




\clearpage



% %%% Local Variables: 
%%% mode: latex
%%% TeX-master: "../main"
%%% End: 
%%% introduction.tex --- 

%% Author: garamonfok@gros
%% Version: $Id: introduction.tex,v 0.0 2011/10/09 18:39:32 garamonfok Exp$

\part{Structure and dynamics of genomes}
\label{part2}


\clearpage
%%% Local Variables: 
%%% mode: latex
%%% TeX-master: "../main"
%%% End: 

\chapter{Life inside genomes, dynamics and predictions}
\label{chap:three}



%%% Local Variables: 
%%% mode: latex
%%% TeX-master: "../main"
%%% End: 

\chapter{Rethinking evolutionary pipelines.}
\label{chap:four}

\clearpage



% %%% Local Variables: 
%%% mode: latex
%%% TeX-master: "../../main"
%%% End: 
%%% introduction.tex --- 

%% Author: garamonfok@gros
%% Version: $Id: introduction.tex,v 0.0 2011/10/09 18:39:32 garamonfok Exp$


% \newpage{}

\bibliographystyle{cc} % serra et al 2011
%  [Serra et al.2011] Francois Serra, Leonardo Arbiza, Joaqu� Dopazo, and Hernan
%  Dopazo, Natural selection on functional modules, a genome-wide analysis. PLoS computa-
%  tional biology 7(3) (2011), p. e1001093.

% \bibliographystyle{apalike} % [serra et al 2011]
%  [Serra et al., 2011] Serra, F., Arbiza, L., Dopazo, J., and Dopazo, H. (2011). Natural
%  selection on functional modules, a genome-wide analysis. PLoS computational biology,
%  7(3):e1001093.

% \bibliographystyle{dcbib}  %[serra 2011]
%  [Serra2011] Fran�ois Serra, Leonardo Arbiza, Joaqu� Dopazo, and Hern�n Dopazo.
%  Natural selection on functional modules, a genome-wide analysis. PLoS com-
%  putational biology, 7(3):e1001093, Natural selection on functional modules,
%  a genome-wide analysis. ISSN 1553-7358. doi:10.1371/journal.pcbi.1001093


% \bibliographystyle{achicago}
%  [Serra et al.2011] Serra, Fran�ois, Leonardo Arbiza, Joaqu� Dopazo, and Hern�n
%  Dopazo. 2011. Natural selection on functional modules, a genome-wide analysis.
%  PLoS computational biology 7 (3): e1001093 (March).




\bibliography{../biblio/bibliography}

\newpage{}
\listoffigures
\listoftables
\end{document}



