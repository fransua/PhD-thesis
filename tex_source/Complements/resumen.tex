%%% resumen.tex --- 

% max 8000 words

\part*{Resumen en castellano}


\section*{Introducción}

En el estudio de las modificaciones genéticas que conducen a las poblaciones a adaptarse a su ambiente, es importante distinguir de forma inequívoca, los cambios que incrementan la eficacia biológica de aquellos que son neutros o levemente deletéreos en un genotipo.

Aunque la neutralidad fue claramente señalada en el mismo ``Origen'' de Charles Darwin, su relevancia dentro del proceso evolutivo fue desestimada hasta el descubrimiento de la enorme variación poblacional detectada en los primeros datos moleculares. Estos descubrimientos condujeron al desarrollo de los modelos neutros de evolución molecular. Gracias a su desarrollo y aceptación, los modelos neutros permitieron formular y probar estadísticamente hipótesis sobre la evolución adaptativa de secuencias biológicas.

En ecología el desarrollo de modelos neutros capaces de entender el patrón observado de diversidad y abundancia de especies es más reciente, y su uso se extendió sólo recientemente, demostrando un ajuste significativo a la casi totalidad de los ecosistemas analizados.

Tanto en ecología como en biología molecular, la definición explícita de los modelos neutros lleva a valorar los cambios adaptativos producidos por selección natural. Más allá de su valor descriptivo, el modelo neutro es también una poderosa herramienta estadística.

Esta tesis analiza tres aproximaciones biológicas diferentes en genomas completos. Cada uno de ellos plantea de forma explícita un modelo neutro y sus desviaciones respectivas. El planteamiento de este común denominador a lo largo de los tres capítulos principales le dan valor explicativo a esta tesis. Sin él, las conclusiones serian irrelevantes. Estas aproximaciones son:
\begin{itemize}
\item \textbf{Contenido de información}: este es  el análisis más simple que se puede hacer de un genoma cómo que entidad contenedora de información biológica. Bajo esta perspectiva, se tiene exclusivamente en cuenta el conjunto de nucleótidos A, C, G y T, considerando estos constituyentes del genoma como unidades fundamentales e independientes. En este estudio consideramos los genomas como una secuencia simple con un determinado contenido informativo. Nuestro primer objetivo es medir el contenido de información de genomas completos. Para ello utilizamos un amplio rango de especies representativo de gran parte de la diversidad de la vida. En el marco de esta tesis, la descripción del contenido de información en genomas se inscribe como el estudio más elemental y seguramente el menos biológico de la descripción del sustrato genómico.
%\item \textbf{Contenido de información}: este es el punto de vista más simple en el análisis de un genoma. Bajo esta perspectiva, se tiene  exclusivamente en cuenta el conjunto de nucleótidos A, C, G y T, considerando estos constituyentes del genoma como unidades fundamentales e independientes. En este estudio  asemejamos los genomas a una secuencia simple con un determinado contenido informativo. Nuestro primer objetivo consiste en definir una estructura común en los genomas. Para ello utilizamos un amplio rango de especies representativo de gran parte de la diversidad de la vida. En el marco de esta tesis, la descripción del contenido de información en  genomas se inscribe como el estudio más elemental y puede que menos biológico de la descripción del sustrato genómico.
\item \textbf{Ecológico}: el estudio de la abundancia y diversidad de familias de elementos genéticos (mayoritariamente elementos transponibles) en genomas eucariotas, si bien está definido desde la genética de poblaciones nunca ha pasado de la modelización de unas pocas de estas familias en un mismo genoma. No obstante los ecólogos ya implementaron modelos estadísticos y mecanísticos capaces de predecir con precisión los patrones y procesos subyacentes a la composición de los ecosistemas. En este contexto, puede parecer natural aplicar estos modelos ecológicos sobre las diferentes familias de elementos genéticos que componen nuestro genoma. De hecho, la ecología del genoma no es un concepto nuevo asociado a esta tesis. Esta segunda aproximación a los genomas completos tiene como fin la elaboración de un modelo general de ecología de genomas para todos los elementos que pueblan los cromosomas eucariotas; y la posterior verificación de este modelo en genomas y cromosomas de un gran número de organismos.
\item \textbf{Sistémico}: esta es la última aproximación bajo la cual analizamos los genomas. Para ello cambiamos el nivel de análisis, pasando de genes individuales a grupos de genes con características funcionales similares. Concretamente, nos centramos en proteínas que actúan conjuntamente para completar una ruta bioquímica o asociadas a una determinada función. En este contexto varios estudios genómicos realizados gen a gen, ya intentaron discernir enriquecimientos funcionales entre los genes sujetos a selección positiva. Sin embargo esta metodología es propensa a perder poder estadístico cuando es aplicada a nivel genómico. En consecuencia, estos estudios no consiguieron resaltar patrones estadísticamente significativos. Nuestro objetivo será implementar una nueva metodología capaz de detectar la huella de la selección natural en diferentes categorías funcionales.
%\item \textbf{Sistémico}: esta será la última perspectiva bajo la cual vamos a analizar los genomas, para ello cambiaremos el nivel de análisis, pasando de genes individuales a grupos de genes con características funcionales  similares. Concretamente, nos centraremos en proteínas que actúan conjuntamente para completar una ruta bioquímica o asociadas a una determinada función. En este contexto varios estudios genómicos realizados gen a gen, ya intentaron discernir enriquecimientos funcionales entre los genes sujetos a selección positiva. Sin embargo esta de metodología es propensa a disminuir el poder estadístico al ser aplicadas a nivel genómico. En consecuencia, estos estudios no consiguieron resaltar patrones estadísticamente significativos. Nuestro objetivo en este estudio será implementar una nueva metodología considerando el genoma en su conjunto a través de la distribución de las variables evolutivas asociadas a cada uno de sus genes. Mediante esta metodología esperamos ser capaces de detectar la huella de la selección natural en diferentes categorías funcionales.
\end {itemize}

El último capítulo de esta tesis constituye la parte mas técnica derivada de cada uno de los tres estudios ya mencionados. Esta parte consiste en el desarrollo de varias herramientas bioinformáticas específicas, entre la cuales se hayan algunas con potencial interés para la comunidad científica. En tres apartados se presenta las siguientes herramientas: Ecolopy, diseñada para el estudio de ecosistemas genómicos; ETE-Evol, para la manipulación de árboles filogenéticos y pruebas de hipótesis evolutivas en regiones codificantes; y Phylemon, un servidor web que propone un amplio abanico de herramientas, todas enmarcadas en los campos de la filogenia, la filogenómica y el test de hipótesis evolutivas.
%Finalmente, dedicamos el último capítulo de esta tesis a la parte mas técnica derivada de estos diferentes estudios realizados a nivel genómico. Esta parte consiste en el desarrollo de varias herramientas bioinformáticas especificas, entre la cuales se hayan algunas con potencial interés para la comunidad científica. En tres apartados presento las siguientes herramientas: Ecolopy diseñada para el estudio de ecosistemas genómicos; ETE-Evol para la manipulación de árboles filogenéticos y el test de hipótesis evolutivas en regiones codificantes; y Phylemon, un servidor web que propone un amplio abanico de herramientas, todas enmarcadas en los campos de la filogenia, la filogenómica y el test de hipótesis evolutivas.

\newpage
\section*{Material y métodos}

\subsection*{Análisis de la complejidad de genomas}

Para el estudio de la complejidad del genoma en términos de cantidad de información, definimos la tasa de complejidad (CR -- por sus siglas en inglés \textit{complexity ratio}). El CR se elabora a partir de tres funciones diferentes que transforman una secuencia de ADN en un valor numérico. Cada una de estas funciones se basa en algoritmos clásicamente utilizados en informática para la compresión de datos. El primero es una transformación de \textit{Burrows-Wheeler}, que ordena los caracteres de una secuencia. Cada carácter se ordena acorde a los caracteres que le siguen en la secuencia. El segundo consiste en aplicar sobre la secuencia previamente ordenada, el algoritmo \textit{Move-To-Front}, que valora numéricamente el desordenamiento que supone cada carácter para la secuencia. Finalmente calculamos la entropía de Shannon correspondiente a los valores derivados del \textit{Move-To-Front}, resultando en un valor de complejidad (CV -- por \textit{complexity value}) que podemos normalizar en base al largo de la secuencia para obtener el CR.

Este análisis fue aplicado sobre 54 genomas cubriendo un amplio rango de formas de vida, desde virus a mamíferos pasando por bacterias, plantas y aves, e incluyendo especies con genomas poliploides, organismos de vida en condiciones extremas, parásitos intracelulares, organismos con expansiones génicas, reducciones genómicas, organismos con genoma de ARN, de una sola hebra, e incluso con un genoma sintético.

%Para el estudio de la complejidad del genoma en términos de cantidad de información definimos la tasa de complejidad (CR -- por sus siglas en inglés \textit{complexity ratio}). El CR consiste en la ejecución de tres pasos para transformar una secuencia de ADN en un valor. Cada uno de estos pasos se basa en algoritmos clásicamente utilizados en informática para la compresión de datos. El primer paso es una transformación de Burrows-Wheeler, que ordena los caracteres de una secuencia, cada carácter se ordena acorde a los caracteres que le siguen en la secuencia. El segundo paso, consiste en aplicar sobre la secuencia previamente ordenada, el algoritmo Move-To-Front, que valora numéricamente el desordenamiento que supone cada carácter para la secuencia. Finalmente calculamos la entropía de Shannon correspondiente a los valores derivados del Move-To-Front, resultando en un valor de complejidad (CV -- por \textit{complexity value}) que podemos normalizar en base a largo de la secuencia para obtener el CR.

%Este algoritmo fue aplicado sobre 54 genomas cubriendo un amplio rango de formas de vida, desde virus a mamíferos pasando por bacterias, planta y aves, e incluyendo especies con genomas típicamente fuera de lo normal como pueden ser los poliploides, organismos de condiciones extremas, parásitos intracelulares, organismos con expansiones génicas, reducciones genómicas, organismos con genoma de ARN, de una sola hebra, o incluso con genoma sintético.

\subsection*{Estudio de la distribución de abundancia de elementos genéticos en eucariotas}

En el contexto de este estudio, usamos 31 genomas de especies eucariotas. El primer paso consistió en identificar los diferentes elementos genéticos y clasificarlos en familias. En este estudio hemos analizado elementos repetidos derivados de la base de datos \textit{RepBase}, y biotipos (o tipos de transcritos, como los codificantes, los ARN-t o los micro-ARN, entre otros) derivados de la bases de datos \textit{Ensembl} de anotación de genomas.

Para poner a prueba la distribución aleatoria de elementos genéticos en cromosomas y genomas hemos simulado su distribución equiprobable y su distribución a través de un proceso neutro donde cada uno de estos elementos está sujeto al principio de equivalencia ecológica.

Para ello hemos desarrollado herramientas estadísticas, inspiradas en las usadas por los ecólogos, a fin de poner a prueba la teoría neutra unificada de biodiversidad (UNTB -- por \textit{Unified Neutral Theory of Biodiversity}) originalmente planteada en ecología por Stephen Hubbell y aplicada en esta tesis para los 548 cromosomas.

%En el contexto de este estudio, usamos los genomas de 31 especies eucariotas. El primer paso consiste en identificar los diferentes elementos genéticos y clasificar los en familias. Además de los elementos repetidos encontrados mediante el programa RepeatMasker, identificamos biotipos (o tipos de transcritos, como los codificantes, los ARN-t o los micro-ARN).

%Para comprobar que la distribución de elementos genéticos sigue patrones aleatorios, simulamos genomas distribuyendo al azar los elementos genéticos originales.

%Desarrollamos una herramienta para poder probar la teoría neutra unificada de biodiversidad (UNTB -- por \textit{Unified Neutral Theory of Biodiversity}) sobre las distribuciones de los diferentes elementos genéticos que identificamos en cromosomas. Esta herramienta, presentada en detalle en la ultima parte de esta tesis, nos permite ajustar por máxima verosimilitud los modelos neutros implementados por Rampal Etienne y Stephen Hubbell, a nuestros datos, concretamente a cada uno de los 548 cromosomas de las 31 especies.

\subsection*{Presiones selectivas a nivel genómico}

El primer paso en el estudio de presiones selectivas en grupos de genes relacionados funcionalmente, es la definición de las especies a analizar y la selección de genes ortólogos. Para este trabajo decidimos concentrarnos en 2 grupos de organismos modelos, los mamíferos con humano, chimpancé, ratón y rata; y \textit{Drosophila} con \textit{D. melanogaster}, \textit{D. sechellia}, \textit{D. simulans}, \textit{D. yakuba} y \textit{D. erecta}). Después de la aplicación de diferentes filtros, identificamos 12.453 y 9.240 grupos de genes ortólogos, respectivamente.

Las presiones selectivas en genes codificantes fueron medidas mediante el valor de $\omega$ ($dN$ sobre $dS$). Además de este cálculo, estimamos el conjunto de genes bajo selección positiva ($\omega > 1$) usando herramientas clásicamente utilizadas en evolución molecular computacional.

En este estudio desarrollamos el ``\textit{Gene Set Selection Analysis}'' (GSSA), una propuesta estadística cuyo propósito es detectar desviaciones significativas en la distribución genómica de una variable evolutiva como los valores $\omega$, $dS$ y $dN$. El GSSA consiste en la aplicación de cinco pasos sucesivos:
\begin{inparaenum}[\bgroup\bfseries\em 1\egroup\it)]
\item ordenar los genes en función de una variable evolutiva,
\item anotar los genes con categorías funcionales,
\item cortar la lista de genes en dos particiones,
\item aplicar un test de Fisher entre los genes anotados con una función dada, y los pertenecientes a una de las particiones definidas, y
\item corregir el conjunto de los p-valores de los resultados por test múltiples.
\end{inparaenum}
Los resultados del GSSA para los diferentes grupos de genes funcionalmente relacionados son de tres tipos:
\begin{inparaenum}[\bgroup\bfseries\em 1\egroup\it)]
\item los no significativos (NS -- por \textit{no-significant}), con ausencia de desviaciones significativas respecto a lo observado en el genoma, y
\item los que muestran una desviación significativa hacia valores altos (SH -- por \textit{significantly high}), o
\item hacia valores bajos (SL -- por \textit{significantly low}) en comparación con lo observado en cada genoma.
\end{inparaenum}
En función de esta observación describimos procesos evolutivos para cada una de las especies y grupos de especies

%Como variables evolutivas probadas a través del GSSA usamos el $\omega$ considerado como indicador directo de las presiones selectivas, las tasas de mutaciones no-sinónimas $dN$, las tasas de mutaciones sinónimas $dS$ (asociadas a cambios neutros) y un valor representativo de las diferencias en presiones selectivas entre un ancestro y su descendiente $\Delta\omega$.


\newpage
\section*{Resultados y discusión}

\subsection*{Estructura cuasi-aleatoria del ADN}

Nuestro primer resultado consiste en la observación de una correlación extraordinaria  entre el valor de complejidad (CV) y el tamaño de la secuencia analizada en 54 especies pertenecientes a los 20 grupos sistemáticos estudiados. Esta co-rrelación es observada en cromosomas y genomas completos (con una pendiente de 0'924 para cromosomas y de 0'967 para genomas). Las tasas de complejidad de información son en su mayoría cercanos al máximo (CR $>$ 0'95). Entre los genomas de menor entropía encontramos, por una parte los poliploides recientes como el maíz, el sorgo o \textit{Danio rerio}, y por otra parte los genomas con una composición de nucleótidos fuertemente sesgada como \textit{Plasmodium} o \textit{Dictyostelium} con concentraciones en A + T superiores al 75\%.

En cuanto a los valores de complejidad en familias de elementos genéticos, observamos que los elementos de mayor complejidad son los genes, y en particular los exones. Por otra parte, en cuanto a los elementos repetidos (donde esperábamos encontrar valores muy bajos de complejidad en información), resultó que las únicas categorías de elementos con valores de complejidad bajo fueron los \myglspl{SINE} y los satélites. Estos resultados apuntan a que los genomas presentan un nivel alto de variabilidad en las regiones con alta concentración de elementos repetidos.

Finalmente, para poder apreciar las diferencias en CR que observamos entre poliploides y no poliploides, simulamos eventos de mutación y de transposición sobre secuencias aleatorias representando genomas y cromosomas poliploides, y también sobre algunos cromosomas de maíz y de sorgo. El resultado de esta simulación, fue que en ambos casos, a través de mutaciones o de transposiciones, el máximo de complejidad se recobró al cabo de un numero de generaciones elevado. En el caso de las mutaciones, usando una tasa de mutación intermedia entre las observaciones para plantas y mamíferos, 30 millones de generaciones fueron suficientes para recobrar un valor de CR $>$ 0'95 para el conjunto de secuencias analizadas.

En este contexto, interpretamos que la evolución del genoma sigue un patrón de sucesivas caídas y crecimientos en cuanto a su CR. Durante este proceso, los estadíos en los que los genomas poliploides recientes sufren mutaciones y reordenamientos, podrían ser propicios para dar origen a nuevas secuencias funcionales, proporcionando así la materia prima de la divergencia entre especies, y el crecimiento de la complejidad biológica.

La conclusión mas destacable es sin duda que, sea cual sea el genoma o el cromosoma analizado, la estructura del ADN esta fuertemente atraída hacia el estado de complejidad máxima. Dejando de lado las excepciones previamente citadas (como los poliploides recientes), observamos que la totalidad de los genomas analizados, desde virus hasta mamíferos, presentan un valor de CR muy cercano a 1.

Generalizando nuestras observaciones podemos formular las siguientes hipótesis:
\begin{itemize}
\item Los genomas presentan una estructura combinatoria cuasi-aleatoria independientemente del grado de complejidad biológico de los organismos.
\item Los genomas poliploides recientes, tienden a recobrar una máxima complejidad a través de procesos de mutación o translocación, después de un número elevado de generaciones.
\item Puesto que la estructura combinatoria del ADN es cuasi-aleatoria, la complejidad del genoma sólo puede aumentar mediante amplificación, y posterior divergencia durante el proceso evolutivo.
\end{itemize}
Nuestras hipótesis podrían verse falseadas si se encontrasen:
\begin{itemize}
\item Poliploides recientes con una estructura de ADN cuasi-aleatoria.
\item No poliploides que muestren una estructura de ADN no aleatoria (CR baja).
\end{itemize}

\subsection*{Diversidad y abundancia de los elementos genéticos en genomas eucariotas}

En primer lugar comparamos la cantidad de los elementos genéticos presentes en cada uno de los 548 cromosomas con lo esperado por una distribución equiprobable de dichos elementos. Sólo un 4\% de los elementos genéticos fueron efectivamente observados en las proporciones esperadas por azar.

Seguimos el estudio con dos metodologías descriptivas clásicas utilizadas en ecología, las curvas de abundancia relativa de especies (RSA -- por \textit{relative species abundances}) que tienen como característica principal representar las especies únicamente acorde a sus abundancias relativas, y la relación entre el número de especies y el tamaño de la área de distribución. Para apoyar la analogía con los estudios de ecología, nos referimos a las diferente familias de elementos genéticos como especies genéticas (GS -- por \textit{genetic species}).

Sorprendentemente las RSA correspondientes a la distribución de GSs, además de presentar una forma muy similar a la observada en ecosistemas, se ajustaban muy bien a lo esperado por azar. El ajuste observado, para el 86\% de los cromosomas estudiados, solo puede explicarse por un proceso balanceado de sobre- y sub-abundancias de cantidad de elementos pertenecientes a diferentes familias en los cromosomas. Esta primera evidencia de ajuste a un patrón ajeno a parámetros biológicos, fue confirmada por la correlación significativa que observamos entre el número de GSs y el tamaño del cromosoma en cuestión.

Ambos resultados evidencian que un proceso aleatorio diferente al proceso de distribución equiprobable rechazado, ajusta la abundancia y diversidad de elementos genéticos en los genomas eucariotas.

El ajuste de los datos al modelo neutro propuesto por la UNTB no pudo ser rechazado para ninguno de los cromosomas analizados (siempre que se aplican las correcciones por múltiples pruebas estadísticas). Este último resultado supone ciertamente la aceptación del principio de equivalencia de cada una de las GSs para explicar el patrón general de abundancias y diversidad en genomas eucariotas.

Si bien el ajuste de un modelo neutro no implica necesariamente la existencia de un proceso neutro responsable del patrón observado, la amplitud del ajuste plantea una pregunta: ¿por qué no somos capaces de detectar el diferencial de la selección natural previamente descrita para la diversidad de familias genéticas en los genomas? Independientemente de la respuesta, el modelo propuesto en este capítulo sirve como hipótesis nula en el estudio de mecanismos alternativos capaces de explicar la abundancia y diversidad de especies genéticas en genomas eucariotas.

\subsection*{Búsqueda de patrones evolutivos en grupos de genes funcionalmente relacionados}

En este capítulo nos dedicamos al estudio de las presiones selectivas en grupos de genes funcionalmente relacionados, adoptando así una escala sistémica para el análisis de los genomas de mamíferos y de \textit{Drosophila}.

Tras la aplicación del GSSA (definido en material y métodos), encontramos muy pocas funciones con sesgos significativos en sus valores $dS$. Sin embargo, para el resto de la variables, encontramos un gran número de resultados positivos, tanto significativamente bajos (SL) como significativamente altos (SH). En gran parte las categorías funcionales que encontramos significativamente aceleradas ($\omega$ SH) coinciden con las tendencias descritas en estudios previos basados en metodologías clásicas de agrupación de genes seleccionados positivamente (PSGs -- por \textit{positively selected genes}). Entre los resultados más destacados podemos mencionar que los módulos funcionales relacionadas con la percepción sensorial presentan valores de $\omega$ alto en primates; o relacionados con inmunidad, también significativamente acelerados, en roedores. En \textit{Drosophila}, encontramos también muchas funciones o rutas metabólicas relacionadas con la percepción sensorial, diferentes metabolismos o proteólisis presentando valores de $\omega$ significativamente altos. De forma general, en mamíferos y en moscas, las rutas metabólicas y funciones moleculares relacionadas con el desarrollo y con la transcripción/traducción resultaron estar muy conservadas (caracterizadas como SL).

Dada la aparente relación entre nuestros resultados y las tendencias encontradas en estudios sobre grupos de PSGs, decidimos relacionar nuestras categorías funcionales con los PSGs, y dividir así nuestros resultados en dos subconjuntos, las categorías funcionales con o sin PSGs. 

Resumiendo el resultado obtenido, los PSGs se distribuyen en módulos funcionales bajo diferentes escenarios evolutivos (con $\omega$ SH, SL e incluso NS), sin embargo, su distribución es significativamente sesgada hacia los grupos funcionales cambiando a tasas elevadas de $\omega$ en roedores y moscas. Por otra parte, en primates, los PSGs parecen distribuirse de manera uniforme entre los módulos funcionales, independientemente de las presiones selectivas observadas en el conjunto de genes asociados a la función.

Esta observación sugiere que los PSGs podrían estar implicados en procesos más complejos que el de su participación directa en los cambios adaptativos de los fenotipos.

A través de la estrategia de evaluación presentada en este capítulo, conseguimos aumentar el poder estadístico en el contexto del análisis de la evolución de genomas y sugerimos que los PSGs podrían cumplir funciones adicionales a la de contribuir a los cambios adaptativos en la evolución de los fenotipos.


\subsection*{Herramientas y programas}

En este último capítulo, se hace referencia a dos herramientas que fueron desarrolladas, en un primer lugar, para responder a necesidades específicas relacionadas con el trabajo presentado en esta tesis y adaptadas, en segundo lugar, para prestar servicio al resto de la comunidad científica. También se presenta el servidor web Phylemon, un recopilatorio de herramientas enmarcadas en la filogenética, la filogenómica y los test de hipótesis evolutivas.

La primera de estas herramientas es Ecolopy, un programa diseñado para estudiar la distribución y abundancias de especies en ecosistemas, y probar estadísticamente su neutralidad mediante modelos enmarcados en la UNTB. Como característica adicional Ecolopy, ofrece la posibilidad de tratar con valores de abundancia muy grandes, como pueden ser los derivados de censos de elementos genéticos. Además del programa de libre acceso a partir del cual se puede llamar las diferentes funciones implementadas, Ecolopy se puede usar a través de un servidor web que integra los principales componentes necesarios para llevar a cabo un test de neutralidad en ecosistemas o genomas.

El segundo programa es una extensión de un paquete de programas llamado ETE, diseñado para tratar con arboles filogenéticos. Esta extensión, ETE-Evol, permite formular y probar un amplio abanico de hipótesis evolutivas, usando internamente programas como CodeML o SLR. ETE-Evol representa sobretodo un avance en el contexto de los estudios genómicos ya que permite enlazar directamente diferentes modelos evolutivos y prueba estadística (por ejemplo el test de selección positiva) a arboles filogenéticos. También resulta útil para el estudio de genes específicos, ya que propone soluciones para representar gráficamente los resultados del cómputo de diferentes modelos evolutivos.

Finalmente, se presenta la segunda versión del servidor web Phylemon. Phylemon nace naturalmente respondiendo a la necesidad de investigadores no-bioinfor-máticos llamados a usar herramientas de uso complejo y asociadas a cómputos pesados; y a investigadores bioinformáticos, intentando alentar el uso de sus herramientas para llegar a un público más amplio de investigadores y estudiantes. Las herramientas propuestas en Phylemon se dividen en la siguientes secciones
\begin{inparaenum}[\bgroup\bfseries\em 1\egroup\it)]
\item Alineamiento: para alinear secuencias,
\item Filogenia: para la construcción de árboles filogenéticos a partir de secuencias alineadas,
\item Pruebas evolutivas: desde las pruebas de ajuste a modelos de substitución de nucleótidos o amino-ácidos hasta pruebas mas complejos como los de selección positiva,
\item ``\textit{Pipeliner}'': una utilidad que permite conectar gráficamente muchas de las herramientas que propone Phylemon formando así un encadenamiento de pasos necesarios, por ejemplo, para pasar de un grupo de secuencias homólogas a la representación de sus relaciones filogenéticas y
\item Utilidades: sección bajo la que se agrupan herramientas accesorias cubriendo un rango de funciones, desde limpiar alineamientos hasta calcular distancias entre árboles filogenéticos.
\end{inparaenum}

\newpage
\section*{Conclusiones}

\begin{enumerate}
  %1
\item A lo largo de toda la diversidad de la vida, desde virus hasta mamíferos, el contenido de información de los genomas muestra valores constantes cercanos al máximo. Sólo los cambios drásticos en el incremento del tamaño del genoma, como pueden ser eventos de poliploidización o sesgos muy evidentes en el contenido de nucleotídicos, son capaces de disminuir el contenido de información del genoma.
  %2
\item Este ajuste universal de los genomas a la máxima complejidad, sugiere que los aumentos en complejidad biológica son la consecuencia de eventos anteriores de expansiones del genoma (mediante duplicación o polyploidización).
  %3 
\item Del mismo modo que para la distribución de especies en ecosistemas, los genomas eucariotas presentan una distribución heterogénea de familias o ``especies'' genéticas: unas pocas son muy abundantes, otras relativamente frecuentes y la mayoría raras.
  %4
\item Al igual que la relación especie-área en ecología, en los genomas eucariotas se observa que el número de especies genéticas es proporcional al tamaño de los cromosomas donde se encuentran.
  %5
\item La distribución y abundancia de las familias de elementos genéticos en genomas eucariotas, ya sea funcional o repetitivo, sigue lo esperado por un modelo neutro similar al desarrollado en la teoría UNTB.
  %6
\item A través del desarrollo y puesta a prueba del GSSA identificamos las principales categorías funcionales candidatas a ser dianas de la selección natural tanto positiva como purificadora durante la evolución de linajes de especies de mamíferos y de \textit{Drosophila}. Dado que el GSSA no está limitado por la presencia obligatoria de genes seleccionados positivamente, la lista de funciones biológicas detectadas como dianas de la selección natural es mayor a las descritas anteriormente.
  %7
\item Los genes bajo selección positiva se distribuyen en categorías funcionales con evidencias significativas de mayor, menor o igual tasa de evolución ($\omega$) que la observada en genomas. Sin embargo se observa un sesgo significativo hacia categorías cambiando a altas tasas de $dN/dS$ en roedores y \textit{Drosophila}. En el caso de primates, los genes seleccionados positivamente se distribuyen de forma uniforme, sugiriendo que, en este caso los tamaños poblacionales afectan la eficacia de la selección natural como se sugiere en la teoría de mutaciones levemente deletéreas.
  %8
\item Dada esta observación sugerimos que el papel de los genes bajo selección positiva no consiste solamente en brindar cambios adaptativos a los fenotipos, sino que posiblemente sirvan para compensar mutaciones deletéreas en una red de genes relacionados funcionalmente.
  %9
\item El trabajo llevado a cabo a lo largo de está tesis condujo al desarrollo de tres herramientas bioinformáticas implementadas con la perspectiva de facilitar y extender futuras investigaciones de la comunidad científica. Estas herramientas se enmarcan en los campos de la ecología (de genomas), la filogenia, la filogenómica y la formulación y prueba de hipótesis evolutivas.
\end{enumerate}

%%% Local Variables: 
%%% mode: latex
%%% TeX-master: "../../thesis_main"
%%% End: 
