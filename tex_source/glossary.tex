%%% glossary.tex --- 

%% Author: francisco@evolution
%% Version: $Id: glossary.tex,v 0.0 2011/10/27 13:42:37 francisco Exp$

\newglossaryentry{seed}{
  name=seed,
  description={\textit{-sequence} of a gene or a protein, is the sequence used as starting point in the search of homologous sequences within a given set of entries. Extending this concept at genomic level, we can talk about \textit{seed-genome} or \textit{seed-species}. \textbf{\em Note:} In a phylome, it is expected to observe an over-representation of proteins belonging from the seed-species},
  plural=seeds
}

\newglossaryentry{ecological niche}{
  name={Ecological niche},
  description={The role of a species of organisms in an ecological community,defined by the resources that the species requires from its environment. The ''competitive exclusion principle'' implies that species can only stably coexist if they have different ecological niches}
}

\newglossaryentry{optimal foraging theory}{
  name={Optimal Foraging Theory},
  description={A theory that is designed to predict the foraging behaviour that maximizes food intake per unit time}
}

\newglossaryentry{selfish DNA}{
  name=Selfish DNA,
  description={Sequences of DNA that accumulate in the genome through non-selective means, and which have a negative effect on the fitnesses of their hosts}
}

\newglossaryentry{SINE}{
  name=SINE Sequence,
  description={A short interspersed element sequence - this is a \gls{retroposon} sequence of less than 500 bp in length that does not encode the protein activities required for its movement},
  plural=SINES
}

\newglossaryentry{LINE}{
  name=LINE Sequence,
  description={A long interspersed element sequence - typically used for non-long terminal repeat retrotransposons},
  plural=LINES
}

\newglossaryentry{retrotransposon}{
  name=Retrotransposon,
  description={An autonomous transposable element that can move to a new location through an RNA intermediate.Long terminal repeat (LTR) retrotransposons have direct repeats of 300-500 bp ofDNA at each end of the element. These sequences resemble the integrated proviruses of retroviruses. Non-LTR retrotransposons lack LTRs and the organization of their coding sequences is more diverged from that of retroviral sequences}
}

\newglossaryentry{retroposon}{
  name=Retroposon,
  description={A mobile DNA sequence that can move to new locations through an RNA intermediate}
}

\newglossaryentry{transposon}{
  name=transposon,
  description={A mobile DNA sequence that moves to new genomic locations through a DNA route, rather than through an RNA intermediate. This movement is catalysed by the action of a transposase protein that is encoded by an autonomous element}
}

\newglossaryentry{trophic}{
  name=trophic,
  description={Of or involving the feeding habits or food relationship of different organisms in a food chain}
}

%%% Local Variables: 
%%% mode: latex
%%% TeX-master: "../master"
%%% End: 
