%%% glossary.tex --- 

%% Author: francisco@evolution
%% Version: $Id: glossary.tex,v 0.0 2011/10/27 13:42:37 francisco Exp$

\newglossaryentry{biotype}{
  name=biotype,
  description={A transcript classification including protein coding, pseudogene, and non-coding RNAs},
}

\newglossaryentry{Burrows-Wheeler}{
  name=Burrows-Wheeler,
  description={\textit{-transform} Also called block-sorting compression, this is an algorithm used in data compression techniques such as bzip2. It is based on the concept of sorting all possible rotations of a given string, sorting the results in lexicographic order and finally taking the last character of each rotated string},
}

\newglossaryentry{C-value}{
  name=C-value,
  description={Refers to the amount of DNA contained within a haploid nucleus. The unit of measurement is picogram (pg)},
}

\newglossaryentry{LINE}{
  name=LINE,
  description={A long interspersed element sequence - typically used for non-long terminal repeat retrotransposons},
}

\newglossaryentry{LTR}{
  name=LTR,
  description={Long Terminal Repeat A kind of retrotransposon with direct repeats of 300-500bp of DNA at each end of the element. These sequences resemble the integrated proviruses of retroviruses},
}

\newglossaryentry{SINE}{
  name=SINE,
  description={A short interspersed element sequence - this is a \mygls{retroposon} sequence of less than 500 bp in length that does not encode the protein activities required for its movement},
}

\newglossaryentry{class}{
  name=class,
  description={In computer science, a class is the description of the characteristics defining an \mygls{object}. Basically the class is what is written in the program while the object is the result of the execution of a class.},
  plural=classes,
}

\newglossaryentry{de Bruijn}{
  name=de Bruijn,
  description={$\sim$\textit{-sequence} A mathematically-defined string of characters with a perfect equal frequency of sub-sequences, i.e.: every possible combination of logarithmic length appears exactly once as a sequence of consecutive symbols},
}

\newglossaryentry{ecological niche}{
  name={ecological niche},
  description={The role of a species of organisms in an ecological community, defined by the resources that the species requires from its environment. The ``competitive exclusion principle'' implies that species can only stably coexist if they have different ecological niches}
}

\newglossaryentry{non-LTR}{
  name=non-LTR,
  description={Non-LTR retrotransposons lack LTRs and the organization of their coding sequences is more diverged from that of retroviral sequences},
}

\newglossaryentry{object}{
  name=object,
  description={In computer science, an object is a symbolic container with its own being defined in a \mygls{class}. An object can incorporate data and methods relating to something of the real world manipulated in a computer program.},
}

\newglossaryentry{instance}{
  name=instance,
  description={In the context of computer science, it is a synonym of an \mygls{object}.},
}

\newglossaryentry{optimal foraging theory}{
  name={Optimal Foraging Theory},
  description={A theory that is designed to predict the foraging behavior that maximizes food intake per unit time}
}

\newglossaryentry{patch}{
  name={patch},
  description={\textit{ecological} $\sim$ A homogeneous area with a given shape and spatial configuration differing from the rest of the ecosystem. It is the lowest unit of a landscape}
}

\newglossaryentry{retroposon}{
  name=retroposon,
  description={A mobile DNA sequence that can move to new locations through an RNA intermediate},
}

\newglossaryentry{retrotransposon}{
  name=retrotransposon,
  description={An autonomous transposable element that can move to a new location through an RNA intermediate. Two major classes of retrotransposons exist, with or without long terminal repeats (see LTR and non-LTR)},
}

\newglossaryentry{satellite}{
  name=satellite,
  description={A kind of \myglspl{tandem repeat}, larger than minisatellites (10-60 bp) and microsatellites (2-6 bp).},
}

\newglossaryentry{script}{
  name=script,
  description={A program written for a software environment that automates the execution of tasks that could alternatively be executed in sequence by a human operator},
}

\newglossaryentry{seed}{
  name=seed,
  description={\textit{-sequence} of a gene or a protein, is the sequence used as a starting point in the search for homologous sequences within a given set of entries. Extending this concept to the genomic level, gives rise to the concepts of \textit{seed-genomes} or \textit{seed-species}. \textbf{\em Note:} in a phylome, it is expected to observe an over-representation of proteins from the seed-species.\\
    $\sim$ \textit{-species}, in the case of ortholog retrieval, a \textit{seed species} is the equivalent of a \textit{seed sequence}},
}

\newglossaryentry{selfish DNA}{
  name=selfish DNA,
  description={Sequences of DNA that accumulate in the genome through non-selective means, and which have a negative effect on the fitness of their hosts}
}

\newglossaryentry{superfamily}{
  name=superfamily,
  description={\textit{Transposable elements'-}, The fourth level in the classification of transposable elements according to \myurl{http://www.bioinformatics.org/wikiposon/doku.php?id=main}. This level classifies elements based on the structure of the internal sequence},
  plural=superfamilies
}

\newglossaryentry{tandem repeat}{
  name=tandem repeat,
  description={Repetitive sequence of DNA, comprising a pattern of two or more sequentially repeated nucleotides.},
}

\newglossaryentry{transposon}{
  name=transposon,
  description={A mobile DNA sequence that moves to new genomic locations through a DNA route, rather than through an RNA intermediate. This movement is catalysed by the action of a transposase protein that is encoded by an autonomous element},
  plural=transposons
}

\newglossaryentry{trophic}{
  name=trophic,
  description={Involving the feeding habits or food relationship of different organisms in a food chain}
}



%%% Local Variables: 
%%% mode: latex
%%% TeX-master: "../../thesis_main"
%%% End: 

